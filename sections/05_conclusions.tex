% We did it.
In this work, we present a novel application of RL to the rich and complex domain of experimental laser physics, using RL as the backbone for a fully automated pulse-shaping routine. Leveraging domain knowledge of the processes regulating phase accumulation in HPL systems, we design a coarse simulator of the pump chain of a HPL system, and we use it to develop control strategies that exclusively use non-destructive measurements in the form of images to maximize the peak intensity of ultra-short laser pulses.

We compare our controller to a representative Bayesian–Optimisation baseline in simulation and observe markedly gentler exploration and peak intensities of up to $90\%$ of the transform‐limited reference. While encouraging, these findings are preliminary and do not yet constitute a comprehensive benchmark against the full suite of gradient–free optimisers. In addition, we reformulate pulse shaping as a Latent MDP and leverage recent domain randomisation techniques to obtain policies that maintain performance under moderate variations of the dynamics parameters.

\paragraph{Limitations.}
% We could have randomized more parameters. the dynamics is also affected by the dispersion coefficients at the compressor.  In particular, the group-delay dispersion coefficient $\gddcomp$ of the compressor directly influences the linear phase accumulated over the spectral profile at compression. Compressor dispersion coefficients mostly depend on the delicate alignment of compressor's optics. As such, one typically assumes not to have control over them, as alignment is a typically cumbersome and lengthy process in ultra-fast optics. Still, being adaptive to changes in their values, particularly for $\gddcomp$ is also paramount to ensure applicability of the policy parameters. Figure~\ref{fig:gddcomp} shows the impact different compression coefficients have on the phase accumulation process, while~\ref{fig:dynamics_peak_intensity} illustrates how the peak intensity of laser pulses varies across values of $B$ and $\gddcomp$.

% Also, the method used is sample inefficient due to it being model-free. Developing a model based algorithm or injecting points of model-based approaches into it might turn out to be a promising approach.

% We leverage strong biases on the processes regulating the dynamics of the system deriving from the advancements on the understanding of laser physics. One could envision using similar biases to learn control policies in a more sample efficient manner.

% We could have modeled fluctuations in B over the course of a single episode instead of assuming it would stay constant

We identify several limitations remaining in our contribution. In particular, HPL systems' performance is known to be influenced, alongside B-integral, by the dispersion coefficients of the compressor. These dispersion coefficients are highly sensitive to the delicate alignment of the compressor optics, which is typically a cumbersome and time-consuming process in ultra-fast optics. As such, we concluded randomizing over these coefficients was unnecessary in a first instance, as a great deal of effort and diagnostic is spent in properly assessing and monitoring the compressor. Still, adapting to their variation as well is a very promising approach, which we seek to investigate further. \newline 
Another limitation is the sample inefficiency of our method, requiring hundreds of thousands to samples to discover well performing policies. We argue this is particularly problematic considering the knowledge available on the process of phase accumulation in linear and non-linear crystals. While our coarse simulator provides a useful tool for model-free learning, the absence of explicit modeling of the dynamics limits data efficiency. Integrating model-based components could significantly improve sample efficiency.

In spite of the above caveats, we believe our study provides a useful proof of concept for the deployment of DRL in HPL systems, highlighting both its practical potential and the key challenges that must be addressed by future work.