In this work, we present a novel application of RL to the rich and complex domain of experimental laser physics, using RL as the backbone for a fully automated pulse-shaping routine. Leveraging domain knowledge of the processes regulating phase accumulation in HPL systems, we design a coarse simulator of the pump chain of a HPL system, and we use it to develop control strategies that exclusively use non-destructive measurements in the form of images to maximize the peak intensity of ultra-short laser pulses.

We compare our controller to a representative Bayesian Optimization baseline in simulation and observe markedly gentler exploration and peak intensities of up to $90\%$ of the transform-limited reference. While encouraging, these findings are preliminary and do not yet constitute a comprehensive benchmark against the full suite of gradient-free optimisers. In addition, we reformulate pulse shaping as a Latent MDP and leverage recent domain randomisation techniques to obtain policies that maintain performance under moderate variations of the dynamics parameters.

\paragraph{Limitations.}
We identify several limitations remaining in our contribution. In particular, HPL systems' performance is known to be influenced, alongside B-integral, by the dispersion coefficients of the compressor. These dispersion coefficients are highly sensitive to the delicate alignment of the compressor optics, which is typically a cumbersome and time-consuming process in ultra-fast optics. As such, we concluded randomizing over these coefficients was unnecessary in a first instance, as a great deal of effort and diagnostic is spent in properly assessing and monitoring the compressor. Still, adapting to their variation as well is a very promising approach, which we seek to investigate further. Further, another limitation is the sample inefficiency of our method, requiring hundreds of thousands to samples to discover well performing policies. We argue this is particularly problematic considering the knowledge available on the process of phase accumulation in linear and non-linear crystals. While our coarse simulator provides a useful tool for model-free learning, the absence of explicit modeling of the dynamics limits data efficiency. Integrating model-based components could significantly improve sample efficiency.

Despite these limitations, our work takes a significant step toward the integration of DRL in HPL systems, providing a framework that is both practical and adaptable to experimental constraints, and prove the effectiveness of the technique in ultra-short laser physics.